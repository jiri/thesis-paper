\chapter{Emulátor}
\label{chap:emulator}

Druhou částí práce je emulátor, který umožní spouštět strojový kód vyprodukovaný překladačem jehož implementace je popsána v sekci \ref{chap:assembler}. Návrh emulátoru je blíže popsaný v sekci \ref{sec:asm-spec}.

\section{Možnosti řešení}

Jelikož je funkcionalita emulátoru úzce omezena na práci s binárními daty a programovou logiku, bude se počet závislostí projektu blížit nule. Zároveň bude důležité, aby knihovnu bylo co nejsnažší integrovat do jiných projektů, což nám malá stopa závislostí usnadní.

\subsection{Rust}

Stejně jako pro překladač je i pro emulátor možné použít jazyk Rust. Jeho výhody a nevýhody jsou blíže popsány v sekci \ref{assembler:rust}. Pro užití při psaní emulátoru jsou pro nás obzvláště relevantní nízkoúrovňové operace, které má Rust jako součást standardní knihovny. Jmenovitě pak detekce přetečení při sčítání a odčítání, nebo snadná konverze dat mezi jejich bytovou reprezentací a kanonickým typem.

Další z výhod je opět zabudovaná funkcionalita pro testování kódu, které je v emulátoru kritické -- chyby při emulaci, při kterých se emulátor může dostat do stavu nekonzistentního s jeho specifikací, jsou obzvlášť pro začátečeníky naprosto neproniknutelné. Je proto kritické takovým chybám zamezit.

Zároveň by bylo možné při testování využít překladače, který je též napsaný v Rustu, jako knihovny použité pro překlad kódu v testech, což by přispělo k jejich přehlednosti.

\subsection{C++}

Druhou variantou je jazyk C++, který je na implementaci této části práce vhodnější. Hlavním důvodem je kompatibilita s aplikacemi, které budou naši knihovnu využívat. Například s grafickou ladící aplikací navrženou v kapitole \ref{chap:debugger}.

K implementaci emulátoru nejsou zapotřebí téměř žádné knihovny, stávají se proto některé nevýhody jazyka C++ popsané v sekci \ref{sec:asm-cpp} irelevantními. Vhod pak přijde extenzivní podpora jazyka C++ pro práci s bitovými poli a binárními soubory. Pro čtení souboru byla použita standardní knihovna \texttt{filesystem}, která je v jazyce dostupná od verze C++17. Náš kód tudíž není možné ve starší verzi jazyka přeložit, avšak v případě potřeby lze použít knihovnu Boost.Filesystem, jejíž rozhraní odpovídá velice blízce knihovně \texttt{filesystem}.

K samotné správě projektu a překladu poté můžeme použít nástroj CMake, popsaný v sekci\ref{sec:asm-cpp}.

\section{Zvolená technologie}

Pro implementaci emulátoru použijeme jazyk C++. Důvodem této volby je primárně snadnější využití knihovny v jiných projektech, jako bude například ladící program popsaný blíže v kapitole \ref{chap:debugger}.

\section{Technické problémy}

Jedním z technických problémů, které nám přineslo užití C++ je detekce přetečení při sčítání dvou bytů. Manuální detekce není nikterak složitá, avšak jak překladačová sada GCC\cite{gcc-overflow}, tak překladač Clang\cite{clang-overflow} disponují zabudovanými funkcemi \texttt{\_\_builtin\_add\_overflow} a \texttt{\_\_builtin\_sub\_overflow}, které návratovou hodnotou indikují, zda při operaci došlo k přetečení, nebo ne.

Dalším problémem pak byla implementace obslužných funkcí pro vstup a výstup. Ve finální verzi projektu jsou realizovány jako mapa typu \mintinline{c++}{std::unordered_map<u8, IoHandler>}. Typ \mintinline{c++}{IoHandler} je potom jednoduchá struktura obsahující dvě funkce -- jednu pro čtení a druhou pro zápis. Výchozí hodnota struktury IoHandler poté splňuje chování procesoru pro nepřirazené pozice v paměti, mohou být tudíž vytvořeny výchozím konstruktorem. Definice struktury je demonstrována v ukázce \ref{fig:io-handler}.

\begin{listing}
\begin{minted}{c++}
#include <functional>

struct IoHandler {
    std::function<u8()> get = []() {
        return 0x00;
    };
    std::function<void(u8)> set = [](u8) {
        return;
    };
};
\end{minted}
\caption{Definice struktury IoHandler}
\label{fig:io-handler}
\end{listing}

Posledním problémem byla potom samotná krokovací funkce emulátoru. V prvotním návrhu byla logika spouštění jednotlivých instrukcí řešena pomocí kolekce lambda funkcí, indexované pomocí operačního kódu instrukce. Díky využití pokročilých metaprogramovacích technik které jazyk C++ umožňuje bylo pak možné detekovat automaticky o jaký druh instrukce se jedná, přečíst automaticky její operandy a předat je přímo lambda funkci, jak je znázorněno v ukázce \ref{fig:cpp-templates}.

\begin{listing}
\begin{minted}{c++}
this->instruction_map[CALL] = [this] (Address addr) {
	this->push_addr(this->pc);
	this->pc = addr;
};

...

auto opcode = this->read_opcode();
this->instruction_map[opcode].execute();
\end{minted}
\caption{Registrace instrukcí pomocí lambda funkcí}
\label{fig:cpp-templates}
\end{listing}

Ačkoliv se jednalo o elegantní řešení, byl kód nakonec přepsán aby používal dlouhý blok \mintinline{c++}{switch}, jehož větve odpovídají jednotlivým instrukcím. Načítání argumentů ze strojového kódu je implementováno pomocnými funkcemi, které automaticky inkrementují registr \texttt{PC}. Toto řešení je sice méně elegantní, ale oproti originálnímu řešení produkuje podstatně jednodušší binární soubor a je tudíž rychlejší.

\section{Testování}
\label{sec:emu-test}

K testování v jazyce C++ můžeme použít například knihovnu Catch\cite{github-catch}. Jedná se o jedinný hlavičkový soubor, který pomocí maker v jazyce C++ implementuje doménově specifický jazyk, ve kterém se testy definují. Knihovna se za nás poté postará o implementaci funkce \texttt{main}, která spustí všechny definované testy. Knihovna také podporuje výbornou integraci s vývojovým prostředím CLion\cite{clion-catch} a výsledky testování jsou potom podrobně dostupné přímo v něm.

\section{Závěr}

Přestože jsme první část práce implementovali v Rustu, zvolili jsme pro emulátor jazyk C++. Tato volba se ukázala jako správná, protože usnadnila integraci do ladícího programu.

Utrpělo však testování. Ačkoliv knihovna Catch usnadnila samotné psaní testů, jejich spouštění spoléhá na existenci cesty ke spustitelnému programu překladače v proměnné prostředí \texttt{ASSEMBLER}. Je tomu tak hlavně proto, aby testy mohly obsahovat zrojový kód, který se později přeloží do strojového kódu a načte překladačem.