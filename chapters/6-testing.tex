\chapter{Testování}

Abychom ověřili, jak dobře se s námi navrženými nástroji funguje, podrobíme je testování s reálnými uživateli. Testování bude probíhat ve třech krocích.

V prvním kroku obeznámíme uživatele s problematikou, návrhem a možnostmi procesoru. Provedeme ho technickou dokumentací, ve které bude moci později vyhledávat patřičné informace. Také mu poskytneme mu počítač, na kterém jsou nainstalovány všechny aplikace potřebné k začátku testu.

V druhém kroku bude uživatel samostatně řešit několik úloh, které jsou navrženy se zvyšující se obtížností, aby uživatele seznámily s vlastnostmi a funkcemi procesoru postupně. Bude mít k dispozici plnou dokumentaci procesoru i aplikacím, které bude během řešení úloh používat. Na jeho otázky budeme odpovídat, avšak budeme se snažit aby co nejvíce problému vyřešil sám.

V třetím kroku provedeme s uživatelem krátkou reflexi, kde zjistíme které úlohy, funkce mikroprocesoru nebo části aplikace pro něj byly problematické. Ověříme, zda jeho řešení úloh je korektní.

\section{Persony}

Při uživatelském testování je užitečné načtrnout si takzvané persony -- archetypy uživatelů, které budou naši aplikaci používat.

\subsection{Ondřej}

Ondřej je absolventem humanitních studií a nikdy neprogramoval. Jeho koníčkem je stavění modelových železnic. Při svém posledním projektu se rozhodl automatizovat zvedání závor podle časovače a chce se proto naučit programovat pro mikrokontrolery. Neví však kde začít a chce proto nejdříve získat základní znalosti, než bude investovat do drahé vývojové sady.

\subsection{Martin}

Martin je zkušený programátor, který začal programovat webové stránky a na vysoké škole se naučil C++. Jeden z povinných předmětů na jeho škole vyučuje základy programování pro mikrokontrolery a jeho vyučující se rozhodl využít naši platformu pro domácí úkoly.

\subsection{David}

David je zkušený programátor, který byl v mládí fascinován starými herními konzolemi. Jeho zájem v tomto oboru ho přivedl k programování pro staré mikroprocesory jako Intel 8086 nebo Zilog Z80 \todo{Odkazy}. Jeho koníčkem je psaní jednoduchých her pro mikroprocesory a naši virtuální platformu se rozhodl vyzkoušet pro svou novou hru.

\section{Výsledky testování}

\blind[2]