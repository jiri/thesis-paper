\begin{conclusion}

V bakalářské práci jsme se zabývali návrhem a implementací překladače a emulátoru pro instrukční sadu jednoduchého mikrokontroleru, kterým bychom mohli nahradit platformu AVR pro naprosté začátečníky.

V první kapitole jsme analyzovali problém, provedli výzkum mezi studenty předmětu BI-SAP na FIT ČVUT a shledali, že současné řešení problému je nedostačující. V další kapitole jsme pak podrobněji analyzovali problém a rozhodli, které problémové body se pokusíme naším řešením ošetřit.

V následujících kapitolách jsme se zabývali postupně návrhem samotného mikroprocesoru, překladače pro jazyk symbolických adres, emulátoru a ve finále ladící aplikace, pomocí které mohou uživatelé naší platformy spouštět a krokovat své programy.

V poslední kapitole jsme potom výsledný produkt otestovali s několika uživateli. Poznatky z testování jsme vyhodnotili, zkonstatovali které části řešení byly úspěšné a načtrtli možnosti, jak aplikaci na základě výsledků testování vylepšit.

Výsledkem této práce je tak dvojice aplikací. První z nich je překladač z jazyka symbolických adres do strojového jazyka. Druhou je potom multiplatformní grafická ladící aplikace, která umožňuje uživatelům spouštět a krokovat programy, během čehož mohou sledovat obsah paměti a registrů. Ačkoli aplikace splňuje zadání a je použitelná, uživatelské testování odhalilo drobné nedostatky. Pro ideální využití aplikace, například při výuce v předmětu BI-SAP, by bylo vhodné tyto nedostatky vyřešit.

\end{conclusion}
