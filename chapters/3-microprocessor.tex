\chapter{Mikroprocesor}

První částí práce je návrh samotného mikroprocesoru. Ten je inspirován platformou AVR, kterou jsme analyzovali v~sekci \ref{sec:spec-avr}. Většina rozhodnutí učiněných při návrhu platformy je tedy inspirována přímo jí.

\section{Instrukční sada}

Instrukční sada je modelována jako podmnožina instrukční sady AVR, avšak upravená pro jednoduchost a větší edukativní hodnotu. Byla odstraněna většina instrukcí, které nejsou kritické pro řešení problémů, nebo jejichž funkcionalita se dá nahradit vícero jinými instrukcemi. Tedy například většina variant instrukcí s~okamžitou hodnotou místo druhého parametru, jako \texttt{addi}, \texttt{andi} nebo \texttt{subi}. Zároveň každý operační znak má pouze jedinou mnemoniku a naopak -- nedochází ke změně operačního znaku podle parametrů, se kterými je daná mnemonika zkombinována. Tím se předejde zbytečným chybám způsobeným překlepy nebo přehlédnutím se při čtení instrukčního manuálu.

Dalším z~hlavních rozdílů je odstranění instrukcí provádějících skoky podle výsledku srovnávacích operací. Instrukce \texttt{cmp} je často vnitřně implementována jako aritmetické odčítání bez zápisu zpět, je proto užitečné chápat, jak porovnání dvou čísel ovlivní příznaky v~procesoru. V~naší instrukční sadě jsou proto podmíněné skoky možné jen podle příznaků. Díky tomu jsou uživatelé nuceni pochopit jak funguje porovnávací instrukce interně a zároveň je díky tomu instrukční sada jednodušší.

Naše platforma též neklade důraz na znaménka čísel uložených v~paměti nebo v~registru. Význam bitů v~registrech je ponechán plně na uživateli platformy. Chybí proto příznak pro znaménko a další vlastnosti procesorů pracujících se zápornými čísly v~dvojkovém doplňku.

\begin{table}[htbp]
\begin{center}
\begin{tabular}{| l | l | l | l |}
\hline
\textbf{Skupina} & \textbf{Operační znak} & \textbf{Mnemonika} \\
\hline
\multirow{9}{*}{Pomocné} & \texttt{0x00} & \texttt{nop} \\ \hline
                         & \texttt{0x02} & \texttt{sleep} \\ \hline
                         & \texttt{0x03} & \texttt{break} \\ \hline
                         & \texttt{0x04} & \texttt{sei} \\ \hline
                         & \texttt{0x05} & \texttt{sec} \\ \hline
                         & \texttt{0x06} & \texttt{sez} \\ \hline
                         & \texttt{0x07} & \texttt{cli} \\ \hline
                         & \texttt{0x08} & \texttt{clc} \\ \hline
                         & \texttt{0x09} & \texttt{clz} \\
\hline
\multirow{11}{*}{Aritmetika} & \texttt{0x10} & \texttt{add} \\ \hline
                             & \texttt{0x11} & \texttt{adc} \\ \hline
                             & \texttt{0x12} & \texttt{sub} \\ \hline
                             & \texttt{0x13} & \texttt{sbc} \\ \hline
                             & \texttt{0x14} & \texttt{inc} \\ \hline
                             & \texttt{0x15} & \texttt{dec} \\ \hline
                             & \texttt{0x16} & \texttt{and} \\ \hline
                             & \texttt{0x17} & \texttt{or} \\ \hline
                             & \texttt{0x18} & \texttt{xor} \\ \hline
                             & \texttt{0x19} & \texttt{cp} \\ \hline
                             & \texttt{0x1A} & \texttt{cpi} \\ \hline
\hline
\multirow{8}{*}{Větvení} & \texttt{0x20} & \texttt{jmp} \\ \hline
                         & \texttt{0x21} & \texttt{call} \\ \hline
                         & \texttt{0x22} & \texttt{ret} \\ \hline
                         & \texttt{0x23} & \texttt{reti} \\ \hline
                         & \texttt{0x24} & \texttt{brc} \\ \hline
                         & \texttt{0x25} & \texttt{brnc} \\ \hline
                         & \texttt{0x26} & \texttt{brz} \\ \hline
                         & \texttt{0x27} & \texttt{brnz} \\
\hline
\multirow{9}{*}{Manipulace s~daty} & \texttt{0x30} & \texttt{mov} \\ \hline
                                   & \texttt{0x31} & \texttt{ldi} \\ \hline
                                   & \texttt{0x32} & \texttt{ld} \\ \hline
                                   & \texttt{0x33} & \texttt{st} \\ \hline
                                   & \texttt{0x34} & \texttt{push} \\ \hline
                                   & \texttt{0x35} & \texttt{pop} \\ \hline
                                   & \texttt{0x36} & \texttt{lpm} \\ \hline
                                   & \texttt{0x3A} & \texttt{in} \\ \hline
                                   & \texttt{0x3B} & \texttt{out} \\
\hline
\end{tabular}
\end{center}
\caption{Přehled instrukční sady}
\label{tbl:instruction-set}
\end{table}

Instrukční sada je detilněji popsána v~příloze \ref{appendix:mcu-manual}.

\section{Registry}

Mikroprocesor disponuje šestnácti víceúčelovými osmibitovými registry \( R0 \) -- \( R15 \). Dále potom speciálními registry \( PC \) a \( SP \), označujícími současnou adresu spouštěné instrukce a vrcholem zásobníku respektive.

V~kontrastu s~platformou AVR jsme snížili počet registrů na šestnáct, abychom mohli kódovat index registru jako čtyřbitovou hodnotu a zvýšili tak čitelnost strojového kódu.

Dále máme dvojici šestnáctibitových registrů $X$ a $Y$, které hodnotou odpovídají dvojicím registru $R12:R13$ a $R14:R15$. Registr $X$ označuje cílovou adresu v~paměti pro zapisovací instrukci \texttt{st}, registr $Y$ potom adresu pro čtecí instrukce \texttt{lpm} a \texttt{ld}.

Zásobník reprezentovaný registrem $SP$ začíná ve výchozím stavu na adrese \texttt{0xFFFF}. Roste směrem k~nižším adresám. Ukazatel zásobníku ukazuje vždy na prázdný prostor nad vrcholem zásobníku, nikoliv na vrchol.

\section{Pamět}

Podobně jako platforma AVR\cite{attiny12-datasheet}, i náš mikroprocesor disponuje hned třemi druhy paměti: programovou, operační a pamětí pro obsluhu vstupu a výstupu.

Tyto tři paměti jsou modelovány jako tři separátní paměťové bloky, a pokrývají své adresní prostory (16, 16 a 8 bitů respektive). V~nižších částech paměti však nejsou namapované registry, tak jak je tomu u~AVR\cite{attiny12-datasheet} a paměť zodpovídající za obsluhu vstupu a výstupu má separátní prostor adres, nezávislý na programové a operační paměti.

Paměť pro obsluhu vstupu a výstupu není fyzickou pamětí, nýbrž adresním prostorem na jehož pozicích jsou namapované obsluhy jednotlivých rozhraní. To je odraženo i v~samotném emulátoru, který ukládá pro každou pozici dvě obslužné funkce. Ty potom obsluhují chování instrukcí \texttt{in} a \texttt{out} pro patřičnou adresu. V~nevyužitém stavu pak zápis do této paměti hodnotu zahazuje a čtení vždy vrací nulovou hodnotu.

\section{Instrukční cyklus}

Mikroprocesor začíná spouštět program od adresy \texttt{0x0000}, na které je umístěný reset vektor. Ještě před načtením adresy dojde k~obsluze přerušení, pokud jsou zapnuta (tj. příznak přerušení je nastaven). Pokud nastalo přerušení, dojde k~zavolání příslušného vektoru přerušení a zakázání přerušení odnastavením jejich příznaku. 

Všechny instrukce probíhají v~jednom cyklu a jsou tedy stejně složité.

\section{Přerušení}

Přerušení jsou při startu spouštění vypnuty a musí se ručně povolit instrukcí \texttt{sei}. Díky tomu je na možné začátku výuky psát kód přímo od počáteční adresy a nestrachovat se, že přepisujeme vektory přerušení. Každý vektor má 16 bytů a jsou pro ně vyhrazeny adresy \texttt{0x0000} -- \texttt{0x0100}. 

\begin{table}[htbp]
\begin{center}
\begin{tabular}{| r | l | l |}
\hline
0x0000 & Reset    & Počáteční bod spouštění \\ \hline
0x0010 & VBlank	  & Indikuje obnovovací frekvenci displeje \\ \hline
0x0020 & Button   & Indikuje, že bylo stisknuto tlačítko \\ \hline
0x0030 & Keyboard & Indikuje, že byl přijat znak z~klávesnice \\ \hline
0x0040 & Serial   & Indikuje, že na sériový port přišla nová data \\
\hline
\end{tabular}
\end{center}
\caption{Vektory přerušení}
\label{tbl:vector-positions}
\end{table}

\section{Rozhraní}

Rozhraní jsou na naší platformě obsluhovány pomocí instrukcí \texttt{in} a \texttt{out}. Změna v~jejich stavu je indikována přerušeními, které provedou před načtením instrukce skok na patřičný vektor, kde můžeme rozhraní obsloužit a vrátit se ke spouštění zbytku programu tak kde jsme přestali pomocí instrukce \texttt{reti}.

Tato část mikroprocesoru se od platformy AVR liší nejvíce. Mikroprocesor nedisponuje žádnými víceúčelovými porty (v~anglické literatuře též známy jako \uv{GPIO pins}). Rozhraní mikroprocesoru jsou dané fixně a rozšiřitelné pouze úpravou programu implementujícího chování pomocí knihovny s~emulátorem.

Mezi podporované rozhraní patří sériový port, který umožňuje textovou komunikaci s~vnějším světem. Dále pak 7 tlačítek, rozložené do pozicových šipek, centrálního potvrzovacího tlačítka a dvou uživatelských tlačítek, jejichž funkce není blíže specifikována. Mikroprocesor podporuje také klávesnicový vstup, který má simulovat přítomnost portu PS/2.

Konečně potom platforma disponuje malým displejem s~osmibitovou paletou a rozměry 160 pixelů na šířku a 144 pixelů na výšku. Displej má obnovovací frekvenci 60Hz a jeho obsah odpovídá jednomu bytu na pixel, počínajíc adresou \texttt{0x8000} v~operační paměti.

Adresy obslužné paměti pro jednotlivá rozhraní a pozice vektorů přerušení lze nalézt v~příloze \ref{appendix:mcu-manual}.
