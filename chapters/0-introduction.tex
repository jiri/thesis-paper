\begin{introduction}

S postupujícím technologickým vývojem je elektronika se kterou se běžně setkáváme čím dál složitější. Instrukční sada procesorů s architekturou x86 od firmy Intel nyní obsahuje tisíce instrukcí\cite{x86-instructions}, což je oproti původnímu procesoru 8086\cite{8086-instructions} závratné číslo. Tento vývoj je ve světle čím dál větší potřeby pro výpočetní výkon nevyhnutelný, ale rostoucí komplexita hardwaru vytváří nepřátelské prostředí pro naprosté začáteníky.

V dobách, kdy nejsofistikovanější domácí počítače jako Amiga nebo Commodore 64 byly poháněny osmibitovými procesory s jednoduchou instrukční sadou, bylo začít s programováním pro danou platformu často pouze otázkou přečtení manuálu, nebo pomoci od učitele ve škole. Dnes je však spouštění aplikací pod kontrolou operačního systému, zavaděč systému je v některých případech zamčený a instrukční sada tak složitá, že obsáhnout ji je pro začátečníka takřka nemožný úkol.

Přesto však existuje zjevný zájem o nízkoúrovňové programování. Populární platforma Arduino se rapidně blíží milionu aktivních uživatelů a díky dobré programové podpoře je nahrávání programů do mikroprocesorů jednodušší, než kdy dřív. Přesto však existuje bariéra, kterou představuje nutnost vlastnit fyzické zařízení schopné daný kód spouštět.

\section{Cíle práce}

V naší práci se pokusíme navrhnout virtuální platformu, která se pokusí být ještě jednodušší, než nejjednodušší z existujících mikroprocesorů a bude možné s ní experimentovat na kterémkoliv moderním domácím počítači.

\section{Struktura práce}

Tato práce je rozdělena do sedmi kapitol. V první kapitole prozkoumáme, s jakými problémy se začátečníci setkávají při programovaní pro mikrokontrolery. Dále se potom seznámíme s existující programovou podporu pro vývoj na platformě AVR. Prozkoumáme možnosti emulace běhu mikrokontrolerů z rodiny AVR na platformě x86.

V druhé kapitole rozvedeme specifikaci aplikačního software, který v rámci práce budeme vytvářet. Stejně jako samotná práce je pak i návrh rozdělen na čtyři části:

\begin{itemize}
	\item architektura mikrokontroleru a instrukční sada,
	\item assembler, tj. aplikace, kterou budeme překládat symbolický zápis na strojový kód,
	\item emulátor, realizovaný formou knihovny,
	\item ladící aplikace, kterou budeme emulovat běh na platformě x86.
\end{itemize}

V následujících čtyřech kapitolách se budeme každé části věnovat detailněji. Vzhledem k tomu, že jsou na sobě jednotlivé části víceméně nezávislé, budeme se věnovat návrhu a realizaci každé z nich pohromadě. Nejdříve zvážíme dílčí technologie, které bychom při realizaci mohli použít a každou z nich detailněji prozkoumáme. Poté si přiblížíme detaily samotné realizace a popíšeme, jak byly řešeny problémy, na které při práci jistě narazíme. Na konci zhodnotíme výsledné řešení a jak se použité technologie osvědčily.

Ve sedmé a poslední kapitole námi navržený aplikační software otestujeme s několika uživateli. Kapitola obsahuje popis metodiky testování, jeho průběh a zhodnocení výsledků. Součástí je také návrh úloh, které při testování použijeme.

\end{introduction}
