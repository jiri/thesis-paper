\begin{introduction}

S postupujícím technologickým vývojem je elektronika se kterou se běžně setkáváme čím dál složitější. Instrukční sada procesorů od firmy Intel\todo{Cite} nyní obsahuje tisíce instrukcí\cite{x86-instructions}, oproti původnímu procesoru 8086\cite{8086-instructions}. Tento vývoj je ve světle čím dál větší potřeby pro výpočetní výkon nevyhnutelný, ale rostoucí komplexita hardwaru vytváří nepřátelské prostředí pro naprosté začáteníky.

V dobách, kdy nejsofistikovanější domácí počítače jako EXAMPLE 1 nebo EXAMPLE 2 byly poháněny osmibitovými procesory s jednoduchou instrukční sadou\todo{cite} bylo začít s programováním pro danou platformu často pouze otázkou přečtení manuálu, nebo pomoci od učitele ve škole. Dnes je však spouštění aplikací pod kontrolou operačního systému, zavaděč systému je zamčený\todo{ref} a instrukční sada tak složitá, že obsáhnout ji je pro začátečníka takřka nemožný úkol.

Přesto však existuje zjevný zájem o nízkoúrovňové programování. Populární platforma Arduino se rapidně blíží milionu aktivních uživatelů a díky dobré programové podpoře je nahrávání programů do mikroprocesorů jednodušší, než kdy dřív. Přesto však existuje bariéra, kterou představuje nutnost vlastnit fyzické zařízení schopné daný kód spouštět.

V naší práci se pokusíme navrhnout virtuální platformu, která se pokusí být ještě jednodušší, než nejjednodušší z existujících mikroprocesorů a bude možné ji použít na kterémkoliv moderním domácím počítači.
\end{introduction}
