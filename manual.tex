\documentclass[a4,12pt]{article}

\usepackage[utf8]{inputenc}
\usepackage[czech]{babel}
\usepackage{titling}
\usepackage{nopageno}
\usepackage{hyperref}

\title{Dokumentace k mikrokontroleru}
\author{Jiří Šebele}
\date{\today}

\newcommand{\instruction}[1]{\subsection{\texttt{#1}}\label{ins:#1}}
\newcommand{\syntax}[1]{\item[Syntaxe:] \texttt{#1} }
\newcommand{\operands}[1]{\item[Operandy:] \( #1 \) }
\newcommand{\operation}[1]{\item[Operace:] \( #1 \) }
\newcommand{\zflag}[1]{\item[Příznak \texttt{Z}:] #1 }
\newcommand{\cflag}[1]{\item[Příznak \texttt{C}:] #1 }
\newcommand{\counter}[1]{\item[Čítač:] \( PC \leftarrow PC + #1 \) }
\newcommand{\stack}[1]{\item[Zásobník:] \( #1 \) }

\newcommand{\reg}[1]{0 \leq #1 \leq 15}
\newcommand{\addr}[1]{0 \leq #1 \leq 65535}
\newcommand{\io}[1]{0 \leq #1 \leq 255}
\newcommand{\byte}[1]{0 \leq #1 \leq 255}

\begin{document}

\maketitle

\section{Instrukční sada}

\instruction{nop}

Neprovede žádnou operaci.

\begin{description}
	\syntax{nop}
	\counter{1}
\end{description}

\instruction{sleep}

Uvede zařízení do režimu spánku.

\begin{description}
	\syntax{sleep}
	\counter{1}
\end{description}

\instruction{break}

V ladícím režimu zastaví emulátor, jinak neprovede žádnou operaci.

\begin{description}
	\syntax{break}
	\counter{1}
\end{description}

\instruction{ei}

Povolí přerušení.

\begin{description}
	\syntax{ei}
	\counter{1}
\end{description}

\instruction{di}

Zakáže přerušení.

\begin{description}
	\syntax{di}
	\counter{1}
\end{description}

\instruction{add}

\begin{description}
	\syntax{add Rd, Rr}
	\operands{\reg{d}, \reg{r}}
	\operation{Rd \leftarrow Rd + Rr}
	\zflag{Nastaven pokud je výsledkem operace 0}
	\cflag{Nastaven pokud došlo k přenosu do vyššího řádu}
	\counter{2}
\end{description}

\instruction{jmp}

Nepodmíněný skok na adresu.

\begin{description}
	\syntax{call A}
	\operands{\addr{A}}
	\item[Čítač:] \( PC \leftarrow A \)
\end{description}

\instruction{call}

Skok do podprogramu. Umístí návratovou adresu na zásobník.

\begin{description}
	\syntax{call A}
	\operands{\addr{A}}
	\operation{STACK \leftarrow PC + 1, PC \leftarrow A}
	\stack{SP \leftarrow SP - 2}
	\counter{3}
\end{description}

\instruction{ret}

Návrat z podprogramu. Návratová adresa je načtena ze zásobníku.

\begin{description}
	\syntax{ret}
	\operation{PC \leftarrow STACK}
	\stack{SP \leftarrow SP + 2}
\end{description}

\instruction{reti}

Stejně jako \texttt{ret} (\ref{ins:ret}), ale nastaví příznak přerušení.

\instruction{brc}

Skočí na adresu, pokud je nastaven příznak \texttt{C}.

\begin{description}
	\syntax{brc A}
	\operands{\addr{A}}
	\item[Čítač:] Pokud \(C\): \( PC \leftarrow A \), jinak \( PC \leftarrow PC + 1 \)
\end{description}

\instruction{brnc}

Skočí na adresu, pokud není nastaven příznak \texttt{C}.

\begin{description}
	\syntax{brnc A}
	\operands{\addr{A}}
	\item[Čítač:] Pokud \(\overline{C}\): \(PC \leftarrow A \), jinak \( PC \leftarrow PC + 1 \)
\end{description}

\instruction{brz}

Skočí na adresu, pokud je nastaven příznak \texttt{Z}.

\begin{description}
	\syntax{brz A}
	\operands{\addr{A}}
	\item[Čítač:] Pokud \(Z\): \(PC \leftarrow A \), jinak \( PC \leftarrow PC + 1 \)
\end{description}

\instruction{brnz}

Skočí na adresu, pokud není nastaven příznak \texttt{Z}.

\begin{description}
	\syntax{brnz A}
	\operands{\addr{A}}
	\item[Čítač:] Pokud \(\overline{Z}\): \(PC \leftarrow A \), jinak \( PC \leftarrow PC + 1 \)
\end{description}

%        /* Arithmetic */
%        map.insert("add",   0x10);
%        map.insert("addc",  0x11);
%        map.insert("sub",   0x12);
%        map.insert("subc",  0x13);
%        map.insert("inc",   0x14);
%        map.insert("dec",   0x15);
%        map.insert("and",   0x16);
%        map.insert("or",    0x17);
%        map.insert("xor",   0x18);
%        map.insert("cmp",   0x19);
%        map.insert("cmpi",  0x1A);
%

\instruction{mov}

Zkopíruje hodnotu z registru do registru.

\begin{description}
	\syntax{mov Rd, Rr}
	\operands{\reg{d}, \reg{r}}
	\operation{Rd \leftarrow Rr}
	\counter{2}
\end{description}

\instruction{ldi}

Načte do registru osmibitovou konstantu.

\begin{description}
	\syntax{ldi Rr, V}
	\operands{\reg{r}, \byte{V}}
	\operation{Rr \leftarrow V}
	\counter{3}
\end{description}

\instruction{ld}

\todo{Doplnit}

\instruction{st}

\todo{Doplnit}

\instruction{push}

Umístí hodnotu z registru na zásobník.

\begin{description}
	\syntax{push Rr}
	\operands{\reg{r}}
	\operation{STACK \leftarrow Rr}
	\stack{SP \leftarrow SP + 1}
	\counter{2}
\end{description}

\instruction{pop}

Umístí hodnotu ze zásobníku do registru

\begin{description}
	\syntax{pop Rr}
	\operands{\reg{r}}
	\operation{Rr \leftarrow STACK}
	\stack{SP \leftarrow SP + 1}
	\counter{2}
\end{description}

\instruction{lpm}

\todo{TODO}

\instruction{ldd}
\instruction{std}
\instruction{lpmd}

\instruction{in}

Načte do registru hodnotu z vstupu / výstupu.

\begin{description}
	\syntax{in Rr, A}
	\operands{\reg{r}, \byte{A}}
	\operation{Rr \leftarrow IO[A]}
	\counter{3}
\end{description}

\instruction{out}

Zkopíruje hodnotu z registru do vstupu / výstupu.

\begin{description}
	\syntax{out Rr, A}
	\operands{\reg{r}, \byte{A}}
	\operation{IO[A] \leftarrow Rr}
	\counter{3}
\end{description}


\end{document}
