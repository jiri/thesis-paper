% arara: pdflatex
% arara: makeglossaries
% arara: biber
% arara: pdflatex
% arara: pdflatex

\documentclass[thesis=B,czech,hidelinks]{template/FITthesis}

\usepackage[utf8]{inputenc}
\usepackage{pdfpages}
\usepackage{graphicx}
\usepackage{dirtree}
\usepackage{minted}
\usepackage{pgfplots}
\usepackage[nonumberlist]{glossaries}

\setlength{\parskip}{1em}

\usepackage{xcolor} 
\newcommand{\todo}[1]{\textcolor{red}{\textbf{[[ #1 ]]}}}

\usepackage{blindtext}
\newcommand{\blind}[1][1]{\textcolor{gray}{\Blindtext[#1][1]}}
%\renewcommand{\blind}[1][1]{#1}

\setlength{\fboxsep}{0.005pt}
\newcommand{\tmpframe}[1]{\fbox{#1}}
%\renewcommand{\tmpframe}[1]{#1}

\newcommand{\todoimage}[1]{%
\begin{figure}%
  \tmpframe{\includegraphics[width=\linewidth]{pdfs/TODO-image}}%
  \caption{#1 \todo{Doplnit obrázek}}%
  \label{fig:TODO}%
\end{figure}%
}

\newcommand{\imagefigurefull}[3]{
\begin{figure}[htbp]
  \centering
  \includegraphics[width=#3\linewidth]{media/#1}
  \caption{#2 \label{pic:#1}}
\end{figure}
}

%\renewcommand{\lstlistingname}{Ukázka}
%\renewcommand{\lstlistlistingname}{Seznam ukázek}

\acknowledgements{Děkuji svému vedoucímu za připomínky a pomoc při psaní práce. Děkuji své rodině a přátelům za podporu. Dále bych rád poděkoval všem respondetům průzkumu a všem účastníkům uživatelského testování. V neposlední řadě bych rád poděkoval své partnerce Arlette a svému příteli Davidovi, kteří mi byli po celou dobu psaní práce inspirací i podporou.}

\abstractEN{This work focuses on the design and implementation of an assembler, emulator and a debugging application for an instruction set of a simple processor, which will enable beginners to orient themselves in the field of programming in the assembly language. It allows constraining of the instruction set for exercises, in which we want to demonstrate specific attributes of assembly programming.
}
\abstractCS{Vytvořte emulátor instrukční sady jednoduchého procesoru, který by mohli používat studenti seznamující se s programováním v asembleru. Instrukční sada má být nastavitelná tak, aby se student s jednotlivými rysy asembleru mohl seznamovat postupně. Instrukčním sada by se měla podobat procesoru AVR, který pak studenti programují. Emulátor musí dovolit základní ladicí úkony, jako je krokování, sledování a modifikace obsahu pamětí.

\todo{Opravit překlepy}}

\title{USB Flash Drive Writer}
\authorGN{Jiří}
\authorFN{Šebele}
\authorWithDegrees{Jiří Šebele}
\author{Jiří Šebele}
\supervisor{doc. Ing. Jan Schmidt, Ph. D.}
\keywordsCS{\todo{keywords}}
\keywordsEN{\todo{keywords}}
\department{Katedra číslicového návrhu}
\placeForDeclarationOfAuthenticity{Prague}
\declarationOfAuthenticityOption{5}
\website{https://github.com/jiri/thesis-paper}
\assignment{assignment.pdf}


\makeglossaries
\newacronym{AVR}{AVR}{Alf (Egil Bogen) and Vegard (Wollan)'s RISC Processor}
\newacronym{GPIO}{GPIO}{General-purpose Input/Output}
\newacronym{RAM}{RAM}{Random-Access Memory}
\newacronym{RISC}{RISC}{Reduced Instruction Set Computing}
\newacronym{SD}{SD}{Secure Digital}
\newacronym{FIT}{FIT}{Fakulta Informačních Technologií}
\glsaddall

\begin{document}

\label{Úvod}

\blind[3]

\chapter{Analýza problému}

\blind[1]

\section{Programování pro mikroprocesory}

\blind[1]

\subsection{Optimalizace}

\blind[1]

\subsection{Hardwarové omezení}

\blind[1]

\section{Výuka}

\blind[2]

\subsection{Složitost mikroprocesoru}

\blind[2]

\subsection{Potřeby reálných nástrojů}

\blind[1]

\section{Existující řešení}

\subsection{Kompilace}

\subsubsection{AVR32.exe}

\subsubsection{Avra}

\subsubsection{GCC}

\subsection{Spouštění}

\subsubsection{Simavr}

\subsubsection{AVR Simulator IDE}

\subsection{Integrované vývojové prostředí}

\subsubsection{Arduino IDE}

\subsubsection{Arduino IDE}

\section{Odůvodnění nového řešení}

\blind[1]

\subsection{Assembly jako jeden z prvních jazyků}

\blind[1]

\subsection{Assembly jako druhý jazyk}

\blind[2]

\subsection{Odstínění reálných HW problémů}

\blind[1]
\chapter{Specifikace a cíle práce}

Po dokončení analýzy současných řešení jsme připraveni definovat které pro\-blé\-my budeme v~práci řešit a které problémy spadají mimo rámec práce.

\section{Využití}

Reálných platforem existuje na trhu hned několik a většina z~nich s~sebou nese i programovou podporu orientovanou na danou platformu. Každá z~platforem je však omezena svým fyzickým návrhem a programová podpora je orientována primárně na profesionální vývoj, nikoliv na výuku a začátečníky. Programová podpora často není multiplatformní, postrádá simulátor nebo není simulátor orientovaný na uživatele, ale strojové testování. Velká část z~nich potom spoléhá na fyzické vývojové sady, které může být pro začátečníky často obtížné vybrat nebo zakoupit.

Cílem práce bude proto navrhnout sadu nástrojů, která není omezena reálnými problémy fyzických platforem a jejíž charakteristiky jsou optimalizovány pro použití začátečníky a výuku, nikoliv profesionální užití v~reálném světě. Cílem platformy je poté naučit začátečníky naprosté základy, s~nimiž se jim poté bude pracovat o~něco lépe. Platforma se také bude podobat reálné platformě AVR, která je v~současné době velmi rozšířená v~amatérské sféře, hlavně díky projektu Arduino.

\subsection{Začátečníci}

Naprostí začátečníci jsou hlavní cílovou skupinou projektu. Instrukční sada je navržena co nejmenší, aby ji začátečníci mohli obsáhnout co nejrychleji a mohli znát všechny nástroje, které mají k~dispozici při řešení problémů. Pro začátečníka bude důležitá multiplatformnost ladící aplikace a překladače, spolu s~faktem, že k~vývoji není potřeba fyzická vývojová sada. Začátečníci by při používání naší platformy měli získat zevrubnou představu, co programování pro mikrokontroler obnáší, v~čem se liší od systémového programování a jak se přistupuje k~řešení problémů v~nízkoúrovňovém jazyce.

\subsection{Pokročilí}

Ačkoliv pokročilí uživatelé nejsou cílovou skupinou našeho projektu, existuje díky modulární architektuře projektu možnost použít námi navržený překladač a emulátor ve vlastních projektech, chceme-li přidat programovatelnost pomocí nízkoúrovňového jazyka. Mezi takové případy užití může spadat například návrh alternativních aplikací pro výuku programování nebo počítačových her. V~neposlední řadě pak není složité adaptovat překladač ani emulátor pro alternativní instrukční sadu.

\section{Funkcionalita}

Výsledkem práce by měla být sada aplikačního softwaru, knihoven a dokumentů popisující kompletní proces vývoje a parametry spouštění kódu na virtuálním procesoru.

\subsection{Analýza architektury AVR}
\label{sec:spec-avr}

Nejdříve prozkoumáme architekturu procesorů z~rodiny AVR. Inspirujeme se mikrokontrolerem ATtiny12, který je jeden z~nejjednodušších AVR mikrokontrolerů. Z~dokumentace ATtiny12 od společnosti Atmel\cite{attiny12-datasheet} zjistíme následující:

\begin{itemize}
	\item mikroprocesor disponuje 32 víceúčelovými, 8-bitovými registry,
	\item mikroprocesor disponuje 1KB programové paměti a 64 byty paměti EEPROM,
	\item na procesoru se nachází 6 tzv. GPIO pinů,
	\item procesor operuje v~řádu jednotek Mhz.
\end{itemize}

Při analýze nemusíme zabíhat do přílišných detailů -- snažíme se ji pouze aproximovat, nikoliv simulovat. Výše uvedené charakteristiky jsou pro nás tudíž směrodatné.

Dále se podíváme na instrukční sadu procesorů AVR. Jedná se o~instrukční sadu typu RISC, disponuje tedy méně, rychlejšími instrukcemi. Instrukce ve své strojové podobě mají délku jednoho nebo více 16-bitových slov. Některé instrukce jsou však redundantní a dají se nahradit kombinací ostatních instrukcí.

\subsection{Mikroprocesor a strojový kód}

První částí práce bude popis podoby virtuálního mikroprocesoru, který naše platforma bude používat. Takový procesor by měl být co nejjednoduší, aby bylo pro naprostého začátečníka obsáhnout všechny informace potřebné k~jeho použití v~co nejmenjším čase. Měl by být podobný rodině mikroprocesorů AVR od firmy Atmel, se kterými se potom začátečník bude setkávat například na předmětech BI-SAP a BI-ARD, nebo při samostatných projektech využívající širokou řadu projektů z~rodiny Arduino\footnote{Více na https://www.arduino.cc/}. Jeho instrukční sada by měla být dostatečně kompaktní na její snadné zapamatování a navržená tak, aby bylo jednoduché kódovat a dekódovat instrukce bez použití manuálu nebo externího programu. Finálně by potom procesor měl umožňovat snadnou adaptaci pro další využití v~jiných projektech, aby nebylo nutné vyvíjet nová řešení a dále tím fragmentovat již tak široké spektrum zařízení.

Mikroprocesor bude založen na modifikované harvardské architektuře, kte\-rá je kombinací harvardské architektury a Von Neumannovy architektury. Bude tedy jednu paměť na program, ze které půjde pouze číst a jednu paměť na data, ze které půjde jak číst, tak do ní zapisovat. Tohle je architektura, kterou používá rodina mikroprocesorů AVR\cite{attiny12-datasheet}.

Samotný mikroprocesor spouští pouze strojový kód a není zatížen pře\-kla\-dem jazykem na vyšší úrovni. Strojový kód je bitovou reprezentací za\-kó\-do\-va\-ných instrukcí. Instrukce pro náš procesor se budou skládat z~jednoho a více bytů. První byte enkóduje vždy pouze typ instrukce, o~který se jedná. Následující byty enkódují parametry dané instrukce. V~samotném strojovém kódu se nijak neodrážejí návěstí definované v~jazyce symbolických adres.

Mikroprocesor by měl také podporovat několik vstupních a výstupních operací, obsluhovaných pomocí oddělené paměti a přerušení:

\begin{itemize}
	\item grafický výstup podobou displeje namapovaného do paměti,
	\item komunikaci po seriálové lince,
	\item omezená sada tlačítek.
\end{itemize}

\subsection{Emulátor}
\label{sec:asm-spec}

Jelikož námi navržený procesor bude pouze virtuální, nebude existovat fyzický integrovaný obvod, který by byl schopný spouštět náš strojový kód. Proto, abychom mohli strojový kód spouštět, vytvoříme emulátor, který bude emulovat běh procesoru na jiné, hostující platformě.

Emulátor bude koncipován formou knihovny, kterou bude možné použít při vývoji dalšího softwaru. Tím se zajistí konzistentní chování napříč více programy, ve kterých bude knihovna použita. Zároveň tak bude velmi snadné naimplementovat novou aplikaci, která je schopna spouštět strojový kód naší virtuální platformy.

Z~toho důvodu však musí emulátor také zprostředkovat snadnou implementaci vstupních a výstupních rozhraní uživatelem knihovny. Požadavky na formu vstupu a výstupu se mohou výrazně lišit dle užití. Například terminálová aplikace bude nejspíše mít seriálovou komunikaci zprostředkovanou pomocí standardního vstupu a výstupu. Grafická aplikace naopak bude nej\-spí\-še poskytovat vlastní seriálovou konzoli. Emulátor by také měl poskytovat mož\-nost snadno rozšířit rozhraní procesoru o~vlastní funkcionalitu.

Součástí projektu bude také rozsáhlá testovací sada , která pomůže zajistit stabilitu fungování programu a kontrolu chyb vznikajících změnami v~kódu. Bude testovat chování každé instrukce samostatně a poté interakce instrukcí mezi sebou.

\subsection{Jazyk}

Jelikož samotný procesor rozumí pouze strojovému kódu, bude další částí práce definice jazyka symbolických adres, který nám umožní zapisovat instrukce pro náš procesor ve formátu čitelném lidmi. Tento jazyk bude definovat mnemoniky k~jednotlivým instrukcím a pseudoinstrukce které nejsou reprezentovány ve strojovém kódu ale ovliňují překlad samotný. Dále pak umožní lepší organizaci programu pomocí návěstí, které umožní pojmenovat konkrétní adresu v~programu. Jazyk bude definován pomocí gramatiky, podle které se bude číst ze zdrojového souboru.

Mezi pseudoinstrukce podporované jazykem by měly patřit následující:

\begin{itemize}
	\item ukládání řetězců a uživatelských dat do programové paměti,
	\item nastavování pozice v~binárním souboru,
	\item vložení obsahu z~jiného souboru.
\end{itemize}

\subsection{Překladač}

Překlad z~jazyka symbolických adres do strojového kódu čitelného pro emulovaný procesor bude zprostředkován konzolovou aplikací.

Výstupem aplikace bude samotný binární soubor se strojovým kódem. Volitelně potom soubor obsahující mapu mezi návěstími a pozicemi v~kódu, která pomůže zpřehlednit dekompilovanou verzi kódu v~ladících nástrojích.

Dále překladač bude umožňovat povolit pouze některé instrukce. To může být nápomocné u~jednoduchých úloh, které v~začátečnících často podněcují použití příliš komplikovaného řešení, nebo u~úloh, kde chceme demonstrovat nějakou vlastnost programování pro mikroprocesory omezením instrukční sady, která by jinak pomohla problém vyřešit příliš jednodušše.

\subsection{Ladicí program}

Abychom usnadnili uživatelům naší platformy lazení programů a odstraňování chyb, bude součástí práce i grafická aplikace, která umožní krokovat chod programu, sledovat za běhu obsah registrů a paměti.

\section{Závěr}

Projekt bude sestávat celkem ze čtyř částí, z~nichž jedna bude spočívat pouze v~návrhu samotné platformy a zbývající tři části budou tvořit programovou podporu pro vývoj programů na námi definované platformě. Části programové podpory budou navrženy do jisté míry nezávisle, aby se v~budoucnu daly snáze upravovat a integrovat s~jinými projekty.

\chapter{Mikroprocesor}

\blind[1]

\section{Instrukce}

\blind[2]

\section{Pamět}

\blind[2]

\section{Rozhraní}

\blind[3]
\chapter{Assembler}

\blind[1]

\section{Možnosti řešení}

\blind[1]

\subsection{C++}

\blind[3]

\subsection{Rust}

\blind[3]

\subsection{Python}

\blind[2]

\section{Zvolená technologie}

\subsection{Rust}

\blind[2]

\section{Technické problémy}

\blind[4]

\section{Testování}

\blind[2]

\section{Závěr}

\blind[1]
\chapter{Emulátor}
\label{chap:emulator}

Druhou částí práce je emulátor, který umožní spouštět strojový kód vyprodukovaný překladačem jehož implementace je popsána v sekci \ref{chap:assembler}. Návrh emulátoru je blíže popsaný v sekci \ref{sec:asm-spec}.

\section{Možnosti řešení}

Jelikož je funkcionalita emulátoru úzce omezena na práci s binárními daty a programovou logiku, bude se počet závislostí projektu blížit nule. Zároveň bude důležité, aby knihovnu bylo co nejsnažší integrovat do jiných projektů, což nám malá stopa závislostí usnadní.

\subsection{Rust}

Stejně jako pro překladač je i pro emulátor možné použít jazyk Rust. Jeho výhody a nevýhody jsou blíže popsány v sekci \ref{assembler:rust}. Pro užití při psaní emulátoru jsou pro nás obzvláště relevantní nízkoúrovňové operace, které má Rust jako součást standardní knihovny. Jmenovitě pak detekce přetečení při sčítání a odčítání, nebo snadná konverze dat mezi jejich bytovou reprezentací a kanonickým typem.

Další z výhod je opět zabudovaná funkcionalita pro testování kódu, které je v emulátoru kritické -- chyby při emulaci, při kterých se emulátor může dostat do stavu nekonzistentního s jeho specifikací, jsou obzvlášť pro začátečeníky naprosto neproniknutelné. Je proto kritické takovým chybám zamezit.

Zároveň by bylo možné při testování využít překladače, který je též napsaný v Rustu, jako knihovny použité pro překlad kódu v testech, což by přispělo k jejich přehlednosti.

\subsection{C++}

Druhou variantou je jazyk C++, který je na implementaci této části práce vhodnější. Hlavním důvodem je kompatibilita s aplikacemi, které budou naši knihovnu využívat. Například s grafickou ladící aplikací navrženou v kapitole \ref{chap:debugger}.

K implementaci emulátoru nejsou zapotřebí téměř žádné knihovny, stávají se proto některé nevýhody jazyka C++ popsané v sekci \ref{sec:asm-cpp} irelevantními. Vhod pak přijde extenzivní podpora jazyka C++ pro práci s bitovými poli a binárními soubory. Pro čtení souboru byla použita standardní knihovna \texttt{filesystem}, která je v jazyce dostupná od verze C++17. Náš kód tudíž není možné ve starší verzi jazyka přeložit, avšak v případě potřeby lze použít knihovnu Boost.Filesystem, jejíž rozhraní odpovídá velice blízce knihovně \texttt{filesystem}.

K samotné správě projektu a překladu poté můžeme použít nástroj CMake, popsaný v sekci\ref{sec:asm-cpp}.

\section{Zvolená technologie}

Pro implementaci emulátoru použijeme jazyk C++. Důvodem této volby je primárně snadnější využití knihovny v jiných projektech, jako bude například ladící program popsaný blíže v kapitole \ref{chap:debugger}.

\section{Technické problémy}

Jedním z technických problémů, které nám přineslo užití C++ je detekce přetečení při sčítání dvou bytů. Manuální detekce není nikterak složitá, avšak jak překladačová sada GCC\cite{gcc-overflow}, tak překladač Clang\cite{clang-overflow} disponují zabudovanými funkcemi \texttt{\_\_builtin\_add\_overflow} a \texttt{\_\_builtin\_sub\_overflow}, které návratovou hodnotou indikují, zda při operaci došlo k přetečení, nebo ne.

Dalším problémem pak byla implementace obslužných funkcí pro vstup a výstup. Ve finální verzi projektu jsou realizovány jako mapa typu \mintinline{c++}{std::unordered_map<u8, IoHandler>}. Typ \mintinline{c++}{IoHandler} je potom jednoduchá struktura obsahující dvě funkce -- jednu pro čtení a druhou pro zápis. Výchozí hodnota struktury IoHandler poté splňuje chování procesoru pro nepřirazené pozice v paměti, mohou být tudíž vytvořeny výchozím konstruktorem. Definice struktury je demonstrována v ukázce \ref{fig:io-handler}.

\begin{listing}
\begin{minted}{c++}
#include <functional>

struct IoHandler {
    std::function<u8()> get = []() {
        return 0x00;
    };
    std::function<void(u8)> set = [](u8) {
        return;
    };
};
\end{minted}
\caption{Definice struktury IoHandler}
\label{fig:io-handler}
\end{listing}

Posledním problémem byla potom samotná krokovací funkce emulátoru. V prvotním návrhu byla logika spouštění jednotlivých instrukcí řešena pomocí kolekce lambda funkcí, indexované pomocí operačního kódu instrukce. Díky využití pokročilých metaprogramovacích technik které jazyk C++ umožňuje bylo pak možné detekovat automaticky o jaký druh instrukce se jedná, přečíst automaticky její operandy a předat je přímo lambda funkci, jak je znázorněno v ukázce \ref{fig:cpp-templates}.

\begin{listing}
\begin{minted}{c++}
this->instruction_map[CALL] = [this] (Address addr) {
	this->push_addr(this->pc);
	this->pc = addr;
};

...

auto opcode = this->read_opcode();
this->instruction_map[opcode].execute();
\end{minted}
\caption{Registrace instrukcí pomocí lambda funkcí}
\label{fig:cpp-templates}
\end{listing}

Ačkoliv se jednalo o elegantní řešení, byl kód nakonec přepsán aby používal dlouhý blok \mintinline{c++}{switch}, jehož větve odpovídají jednotlivým instrukcím. Načítání argumentů ze strojového kódu je implementováno pomocnými funkcemi, které automaticky inkrementují registr \texttt{PC}. Toto řešení je sice méně elegantní, ale oproti originálnímu řešení produkuje podstatně jednodušší binární soubor a je tudíž rychlejší.

\section{Testování}
\label{sec:emu-test}

K testování v jazyce C++ můžeme použít například knihovnu Catch\cite{github-catch}. Jedná se o jedinný hlavičkový soubor, který pomocí maker v jazyce C++ implementuje doménově specifický jazyk, ve kterém se testy definují. Knihovna se za nás poté postará o implementaci funkce \texttt{main}, která spustí všechny definované testy. Knihovna také podporuje výbornou integraci s vývojovým prostředím CLion\cite{clion-catch} a výsledky testování jsou potom podrobně dostupné přímo v něm.

\section{Závěr}

Přestože jsme první část práce implementovali v Rustu, zvolili jsme pro emulátor jazyk C++. Tato volba se ukázala jako správná, protože usnadnila integraci do ladícího programu.

Utrpělo však testování. Ačkoliv knihovna Catch usnadnila samotné psaní testů, jejich spouštění spoléhá na existenci cesty ke spustitelnému programu překladače v proměnné prostředí \texttt{ASSEMBLER}. Je tomu tak hlavně proto, aby testy mohly obsahovat zrojový kód, který se později přeloží do strojového kódu a načte překladačem.
\chapter{Ladící program}

\blind[2]

\section{Možnosti řešení}

\blind[1]

\subsection{GTK+}

\blind[1]

\subsection{Qt}

\blind[2]

\subsection{ImGui}

\blind[3]

\section{Zvolená technologie}

\subsection{C++}

\blind[1]

\subsection{Qt}

\blind[2]

\section{Technické problémy}

\blind[4]

\section{Závěr}

\blind[1]
\chapter{Uživatelské testování}

Abychom ověřili, jak dobře se s námi navrženými nástroji funguje, podrobíme je testování s reálnými uživateli. V této kapitole budeme tedy testovat práci jako celek v ohledu používání uživatelem. Testování spolehlivosti překladače a emulátoru jsme již provedli v kapitolách \ref{sec:asm-test} a \ref{sec:emu-test} respektive.

\section{Metodika}

Testování bude probíhat ve třech krocích a bude časově omezeno na hodinu od začátku řešení úloh (tj. nevčetně přípravy).

V prvním kroku obeznámíme uživatele s problematikou, návrhem a možnostmi procesoru. Provedeme ho technickou dokumentací, ve které bude moci později vyhledávat patřičné informace. Také mu poskytneme mu počítač, na kterém jsou nainstalovány všechny aplikace potřebné k začátku testu.

V druhém kroku bude uživatel samostatně řešit několik úloh, které jsou navrženy se zvyšující se obtížností, aby uživatele seznámily s vlastnostmi a funkcemi procesoru postupně. Bude mít k dispozici plnou dokumentaci procesoru i aplikacím, které bude během řešení úloh používat. Na jeho otázky budeme odpovídat, avšak budeme se snažit aby co nejvíce problému vyřešil bez naší pomoci.

V třetím kroku provedeme s uživatelem krátkou reflexi, kde zjistíme které úlohy, funkce mikroprocesoru nebo části aplikace pro něj byly problematické. Ověříme, zda jeho řešení úloh je korektní a zda plně pochopil koncepty, které úlohy měly uživateli představit.

\section{Persony}

Při uživatelském testování je užitečné načtrnout si takzvané persony -- archetypy uživatelů, které budou naši aplikaci používat.

\subsection{Ondřej}
\label{persona:a}

Ondřej je absolventem humanitních studií a nikdy neprogramoval. Jeho koníčkem je stavění modelových železnic. Při svém posledním projektu se rozhodl automatizovat zvedání závor podle časovače a chce se proto naučit programovat pro mikrokontrolery. Neví však kde začít a chce proto nejdříve získat základní znalosti, než bude investovat do drahé vývojové sady.

Jeho definující vlastnosti jsou neznalost základních programovacích konceptů a orientace na externí rozhraní, která plánuje používat.

\subsection{Martin}
\label{persona:b}

Martin je zkušený programátor, který začal programovat webové stránky a na vysoké škole se naučil C++. Jeden z povinných předmětů na jeho škole vyučuje základy programování pro mikrokontrolery a jeho vyučující se rozhodl využít naši platformu pro domácí úkoly.

Mezi jeho definující vlastnosti patří spoléhání na komfort vysokoúrovňových jazyků a dobrá znalost programovacích konceptů.

\subsection{David}
\label{persona:c}

David je zkušený programátor, který byl v mládí fascinován starými herními konzolemi. Jeho zájem v tomto oboru ho přivedl k programování pro staré počítače se kterými se setkával v dětství. Jeho koníčkem je psaní jednoduchých her pro mikroprocesory a naši virtuální platformu se rozhodl vyzkoušet pro svou novou hru.

Mezi jeho definující vlastnosti patří vysoká náročnost na funkcionalitu překladače a kvalitu ladící aplikace.

\section{Úlohy}

Při testování aplikace s uživateli je užitečné mít konzistentní sadu úloh, kterou budou uživatelé při testování řešit. Sada nemusí nutně pokrýt celou funkcionalitu aplikace, ale měla by pokrýt funkcionalitu relevantní pro všechny persony. Zároveň by sada měla rozsahově odpovídat délce testování, pro které je v našem případě vyhrazena jedna hodina. Samotný uživatel potom nemusí úlohy splnit všechny. Počet vyřešených úloh je sám o sobě užitečným indikátorem přívětivosti naší aplikace.

Proto pro uživatelské testování aplikace bylo navrženo šest úloh. Jsou navrženy se vzrůstající obtížností a pokrývají základní operace v jazyce symbolických adres, základní stavební bloky programu a užívání rozhraní pro vstup a výstup. Některé úlohy používají možnost omezit instrukční sadu přijímanou překladačem, aby omezili prostor možných řešení pro danou úlohu.

Samotné zadání úloh je možné najít v příloze \ref{chap:excercises}.

\section{Výsledky testování}

Výsledky testování se výrazně lišily dle persony pod kterou uživatel náleží, jsou proto rozděleny do skupin podle patřičné persony.

\subsection{Persona Ondřej}

Uživatelé odpovídající personě Ondřej (definované v \ref{persona:a}) se nejvíce potýkali se samotným návrhem logiky programu. Jejich prvotní návrhy byly povětšinou deklarativní a procedurální způsob myšlení nad problémy začali používat až po drobných radách. Z testování vyplynulo, že takové přemýšlení je nejefektivnější jim připodobnit k zadáváním příkazů robotovi nebo operacím na výrobní lince. Práce s pamětí a adresace byla potom nejlépe vysvětlena připodobněním k tabulkovému dokumentu.

Po překonání prvotního ostychu, vysvětlení základních programovacích pojmů a nastínění způsobu, kterým je dobré problémy řešit však naprostí začátečníci řešili úlohy samostatně a bez větších nesnází. Reflexe s nimi ukazuje, že tomu tak je hlavně proto, že samotný jazyk symbolických adres je jednoduché obsáhnout jako celek. Nezkušenost, která byla na začátku testování hendikepem, je nyní osvobozovala od tápání po chybějící funkcionalitě z vysokoúrovňových jazyků a umožnila jim lépe se soustředit na řešení samotného problému. Také možnost překladače omezit přijímanou instrukční sadu pomohla rychleji navést uživatele na správnou cestu. Přesto však začátečníci splnili během vymezené hodiny menší množství úloh. Řešení úloh také kvalitativně strádalo oproti účastníkům, kteří mají s programováním předchozí zkušenosti a často se v řešení nacházely neošetřené mezní případy.

Výsledky testování se začátečníky byli až na drobné vyjímky veskrze pozitivní. Samotní uživatelé potom cítili dobrý pocit z odvedené práce a z pokroku, který během hodiny učinili. Část z nich projevila zájem věnovat se programování i po konci testu, většinou k dokončení rozpracované úlohy.

\subsection{Persona Martin}

\blind[2]

\subsection{Persona David}

\blind[1]

\todo{TABULKA S TESTY}

\begin{conclusion}

\blind[2]

\end{conclusion}


\bibliographystyle{csn690}
\bibliography{bibliography}

\printglossary[title=Seznam použitých zkratek,type=\acronymtype]

\appendix

\chapter{Úlohy k testování}
\label{chap:excercises}
\includepdf[pages={1}]{excercises}

\chapter{Dokumentace mikroprocesoru a instrukční sady}
\todo{TODO}

\chapter{Obsah přiložené SD karty}

\begin{figure}
	\dirtree{%
		.1 readme.txt\DTcomment{stručný popis obsahu SD karty}.
		.1 src.
		.2 assembler\DTcomment{zdrojové kódy překladače}.
		.2 emultor\DTcomment{zdrojové kódy emulátoru}.
		.2 debugger\DTcomment{zdrojové kódy ladící aplikace}.
		.2 paper\DTcomment{zdrojová forma práce ve formátu \LaTeX{}}.
		.1 text\DTcomment{text práce}.
		.2 thesis.pdf\DTcomment{text práce ve formátu PDF}.
	}
\end{figure}

\end{document}
