\documentclass[a4,12pt]{exam}

\usepackage[utf8]{inputenc}
\usepackage[czech]{babel}
\usepackage{titling}
\usepackage{nopageno}

\title{Úlohy pro testování}

\preauthor{}
\author{}
\postauthor{}

\predate{}
\date{}
\postdate{}

\qformat{\textbf{Úloha \thequestion:} \thequestiontitle\hfill\thepoints}

\begin{document}

\maketitle

\begin{questions}
	
	\titledquestion{Rovnice}
	
	Napište program, který řeší následující rovnici a oveřte správnost fungování programu na hodnotách \( R1 := 20, R2 := 11 \).
	
	\[ R0 = R1 + 2 \times R2 \]
	
	\titledquestion{Ahoj, světe!}
	
	Nastudujte z dokumentace procesoru, jak funguje sériový vstup a výstup. Napište program, který na seriálový výstup vypíše řetězec ``Ahoj, svete!''. Řetězec by měl být uložen v programové paměti procesoru.
	
	\titledquestion{Zásobník}
	
	Použijte zásobník k prohození obsahu registrů \(R0\) a \(R1\).
	
	\textit{Povolené instrukce:} \textbf{ldi}, \textbf{push}, \textbf{pop}
	
	\titledquestion{Podprogramy}
	
	Napište podprogram, který ze zásobníku přečte postupně vyšší byte a nižší byte adresy řetězce v operační paměti programu. Tento řetězec poté převeďte tak, aby na konci podprogramu byl psán pouze velkými písmeny. Nepísmenné znaky by však měly zůstat nezměněny.
	
	\textit{Rada:} Ke změně malého písmena na velké stačí odsnatavit 6. bit v jeho reprezentaci.
	
	\titledquestion{Vstup a výstup}
	
	Nastudujte z dokumentace procesoru, jak funguje seriálový vstup a výstup.
	
	\begin{parts}
		\part Napište program, který do adresy pro seriálový vstup a výstup zapíše hodnotu \( 42 \) a ihned ji zase přečte do registru \( R0 \). Zkuste předpovědět obsah registru \( R0 \) a porovnejte své očekávání se skutečným výsledkem.
		
		\part Napište program, který každou příchozí zprávu po seriálové lince přepošle ihned zpět. Procesor by v odbobí nečinnosti neměl provádět žádnou práci.
	\end{parts}

	\titledquestion{Grafický výstup}
	
	Nastudujte z dokumentace procesoru, jak funguje grafický výstup procesoru. Napište program s co nejzajímavějším grafickým výstupem -- fantazii se meze nekladou. Změna grafického výstupu by měla být synchronizována s obnovovací frekvencí displeje.
	
\end{questions}

\end{document}
